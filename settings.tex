\documentclass[a4paper,11pt]{article}
\usepackage[utf8]{inputenc}            % Tekenkodning
\usepackage[T1]{fontenc}               % Fixa kopiering av texten
\usepackage[english]{babel}            % Språk (t.ex. Innehåll)
\usepackage{geometry}                  % Sidlayout m.m.
\usepackage{graphicx,epstopdf,float}   % Bilder
\usepackage{amsmath,amssymb,amsfonts}  % Matematik
\usepackage{enumerate}                 % Fler typer av listor
\usepackage{fancyhdr}                  % Sidhuvud/sidfot
\usepackage{hyperref}                  % Hyperlänkar
\usepackage{parskip}                   % noindent!!
\usepackage{float}                     % \begin{figure}[H] preciserar bildposition
\usepackage{subcaption}             % För att lägga figurer bredvidvarandra från: http://tex.stackexchange.com/questions/91224/placing-two-figures-side-by-side
\usepackage{textcomp}
\usepackage{gensymb}
\usepackage{wrapfig}
\usepackage[]{algorithm2e}
% packages for matlab codes
\usepackage{listings}
\usepackage{color}
\usepackage{pdflscape}
\usepackage{multicol}
\setlength{\columnsep}{0.6cm}
% Add new commands used-defined:
\newcommand{\m}[1]{\mathbf{#1}}
\newcommand{\tn}[1]{\textnormal{#1}}
\newcommand{\ve}[1]{\textnormal{vec}(#1)}


% instälningar för figurtexter
%\usepackage[margin=3ex,font=small,labelfont=bf,labelsep=endash]{caption}
\usepackage[font={small,it}]{caption}
\usepackage[labelfont={normal,bf}]{caption}
\usepackage[margin=3ex]{caption}

% mailadresser som hyperlänkar
\newcommand{\mail}[1]{\href{mailto:#1}{\nolinkurl{#1}}}
% Spara författare och titel
\let\oldAuthor\author
\renewcommand{\author}[1]{\newcommand{\myAuthor}{#1}\oldAuthor{#1}}
\let\oldTitle\title
\renewcommand{\title}[1]{\newcommand{\myTitle}{#1}\oldTitle{#1}}

% Hyperlänkar
\hypersetup{
  colorlinks   = true, %Colours links instead of ugly boxes
  urlcolor     = black, %Colour for external hyperlinks
  linkcolor    = black, %Colour of internal links
  citecolor   = black  %Colour of citations
}

\definecolor{mygreen}{rgb}{0,0.6,0}
\definecolor{mygray}{rgb}{0.5,0.5,0.5}
\definecolor{mymauve}{rgb}{0.58,0,0.82}

\lstset{ %
  backgroundcolor=\color{white},   % choose the background color; you must add \usepackage{color} or \usepackage{xcolor}
  basicstyle=\footnotesize\tt,        % the size of the fonts that are used for the code
  breakatwhitespace=false,         % sets if automatic breaks should only happen at whitespace
  breaklines=true,                 % sets automatic line breaking
  captionpos=t,                    % sets the caption-position to bottom
  commentstyle=\color{mygreen},    % comment style
  deletekeywords={...},            % if you want to delete keywords from the given language
  escapeinside={\%*}{*)},          % if you want to add LaTeX within your code
  extendedchars=true,              % lets you use non-ASCII characters; for 8-bits encodings only, does not work with UTF-8
  frame=single,	                   % adds a frame around the code
  keepspaces=true,                 % keeps spaces in text, useful for keeping indentation of code (possibly needs columns=flexible)
  keywordstyle=\color{blue},       % keyword style
  language=matlab,                 % the language of the code
  otherkeywords={...},           % if you want to add more keywords to the set
  numbers=left,                    % where to put the line-numbers; possible values are (none, left, right)
  numbersep=5pt,                   % how far the line-numbers are from the code
  numberstyle=\tiny\color{mygray}, % the style that is used for the line-numbers
  rulecolor=\color{black},         % if not set, the frame-color may be changed on line-breaks within not-black text (e.g., comments (green here))
  showspaces=false,                % show spaces everywhere adding particular underscores; it overrides 'showstringspaces'
  showstringspaces=false,          % underline spaces within strings only
  showtabs=false,                  % show tabs within strings adding particular underscores
  stepnumber=2,                    % the step between two line-numbers. If it's 1, each line will be numbered
  stringstyle=\color{mymauve},     % string literal style
  tabsize=2,	                   % sets default tabsize to 2 spaces
  title=\lstname                   % show the filename of files included with \lstinputlisting; also try caption instead of title
}



\graphicspath{{./matlab/},{./images/},{./matlab/figs/}} % Söker också bilder i en undermapp figs.
