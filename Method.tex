\section{Method}
\label{sec:method of stuff}

In order to study the quantum particle using the methods in Sec. \ref{sec:Theory} the program called \verb|wavepacket| was used \cite{wavepacket}. First, the initial wavefunction was implemented together with a zero potential and the wave was simulated for $6.8 \cdot 10^{-3}$s using $k_0 = 200$ and the resulting wave was analyzed. Next, the potential was replaced with a potential well. For this case simulations was done for different inital wavenumbers from 250 to 350 in steps of 10 in order to find a relation of the reflection and transmission coefficients and thus be able to verify Eq. \eqref{eq:well-T}. The reflection and transmission coefficients was computed as

\begin{equation}
  \label{eq:RTnumerical}
  \begin{split}
    R &= \int_{-\infty}^{-a}|\psi(x,t)|^2\dn x \approx 0.5\psi(x=-8,t) + \sum_{i=1}^{x(i)<-a} \psi(x=-8+i\Delta x,t) + 0.5\psi(x=-a,t)\\
    T &= \int_{a}^{\infty}|\psi(x,t)|^2\dn x \approx 0.5\psi(x=a,t) + \sum_{x(i)>a}^{x(i)<8} \psi(x=a+i\Delta x,t) + 0.5\psi(x=8,t)\\
  \end{split}
\end{equation}

The final coefficients of each simulation was then used to plot $R,T$ against the wavenumber.

The same study was then done with the potential changed to a barrier instead of a well.

As a final study the numerical algorithm was changed and the program \verb|diffScheme| was used. The same results as with the potential barrier was produced and the two algorithms was compared by the means of excecution time. In order to make a fair comparison both programs produced the same output and was set to the same simulated time in means of seconds.

\subsection{wavepacket.c}

The program \verb|wavepacket| is a simulation program for quantum particles using the split-operator method described in Sec. \ref{sec:splitOp}. The program was used with different potentials yielding different executable files. There three potentials can be compiled using either of the commands

\begin{itemize}
  \item \verb|make freepart|
  \item \verb|make potwell|
  \item \verb|make barrier|
\end{itemize}

for the free particle, potential well and potential barrier correspondingly. The program uses \verb|FFTW| in order to compute the fourier transforms so this library must be installed in order to compile the program correctly.

The simulations are then executed by adding and argument file containing the parameters (preferrably the template file in the appendix \ref{App:input}) and an argument with the initial wavenumber. If the input file is found in a folder called \verb|input_files| then a simulation for the potential well using $k_0=200$ could be excetuted using the line

\verb|./potwell input_files/pot_well_template.in 200|

\subsection{waveDiff.c}
\label{sec:diffScheme}

The program implementing the second-order differencing scheme is called \verb|waveDiff| and can be compiled by the command \verb|make diffScheme|. No external libraries was used in the implementation of this code. The same datatypes and general funcitons (e.g. the routine for reading parameters and printing results) as in \verb|wavepacket| was used in order to be able to compare the computation times. A similar command can be used to execute a simulation using this program. Note however that the time step used must be about $5\cdot 10^{-8}$ in order to achieve convergence.
