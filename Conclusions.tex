\section{Conclusion}

As a conclusion one can say that the numerical results for simulating a quantum particle performs well in both predicting the dynamics of it and verifying quantum effects which are otherwise only proven analytically for simple cases and experimentally. The results showed that the wavepacket can be reflected when meeting a potential well and, by resonance, it can also be almost perfectly transmitted. The numerical transmission coefficient matched the analytical expression quite well  but undershot the peak values in transmission. This is due to that the analytical expression is basen on a monochromatic wave and the simulated wave was a wavepacket consisting of waves with a finite bandwidth. Similar results was obtained for the potential barrier in which also the tunneling effect could be proven. This means that a quantum particle meeting a more energetic potetial barrier might still pass through it.

When comparing the two algorithms split operator method and second-prder differencing method the split operator method was a lot faster with respect to computational time. If this result is perfectly valid is not clear since there might be some code optimization still possible to do on the second-order differencing code. In order to be able to draw more conclusions regarding this one must first compare different datatypes and try to make the code more efficient. This report showed however that the split operator method is more likely to be faster.
