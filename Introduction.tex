\section{Introduction}

Quantum particles does not behave like ordinary particles. This is due to the fact that their motion cannot simply be determined by the position and velocity of the particle, that is, because their position is not well defined. According to the Heisenberg principle, it is impossible to determine the exact position and momentum of a particle at the same time. The position can however be modeled by a probability density, spanning all over space. This makes it possible only to find analytic solutions to the dynamics of a quantum particle for only a few sets of potentials and initial conditions. This is why numerical methods are of high importance when studiyng quantum effects. It is possible to model and simulate these by solving the time dependent Schrödinger equation (TDSE) numerically using high level algorithms. The complexity of this situation is that since the position is described by a probability density, a wavefunction, one must store and manipulate data from all over space in each time step in comparison to ordinary particles where only the position an momentum is needed. This makes the problem of solving TDSE non-trivial and only high performance computers are able to finish the task within reasonable time.

In this report, TDSE will be solved for a free particle, a particle hitting a potential well and a potential barrier and the quantum effects that occur will be studied. The reflection and transmission coefficients for these cases will be computed and compared to analytical expressions and conclusions will be drawn for their validity. Two different algorithms will be implemented for solving TDSE and at the end of the report, a short study comparing their performance in means of computational time will be done.

First follows some theory about quantum mechanincs with and emphasis on how TDSE can be solved using two different approaches. Then some analytical results will be presented followed by how the algorithms has been implemented. Then follows the numerical results and analysis.
