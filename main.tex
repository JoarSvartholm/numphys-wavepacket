\documentclass[a4paper,11pt]{article}
\usepackage[utf8]{inputenc}            % Tekenkodning
\usepackage[T1]{fontenc}               % Fixa kopiering av texten
\usepackage[english]{babel}            % Språk (t.ex. Innehåll)
\usepackage{geometry}                  % Sidlayout m.m.
\usepackage{graphicx,epstopdf,float}   % Bilder
\usepackage{amsmath,amssymb,amsfonts}  % Matematik
\usepackage{enumerate}                 % Fler typer av listor
\usepackage{fancyhdr}                  % Sidhuvud/sidfot
\usepackage{hyperref}                  % Hyperlänkar
\usepackage{parskip}                   % noindent!!
\usepackage{float}                     % \begin{figure}[H] preciserar bildposition
\usepackage{subcaption}             % För att lägga figurer bredvidvarandra från: http://tex.stackexchange.com/questions/91224/placing-two-figures-side-by-side
\usepackage{textcomp}
\usepackage{gensymb}
\usepackage{wrapfig}
\usepackage[]{algorithm2e}
% packages for matlab codes
\usepackage{listings}
\usepackage{color}
\usepackage{pdflscape}
\usepackage[nottoc]{tocbibind}
\usepackage{multicol}
\setlength{\columnsep}{0.6cm}
% Add new commands used-defined:
\newcommand{\m}[1]{\mathbf{#1}}
\newcommand{\tn}[1]{\textnormal{#1}}
\newcommand{\dn}{\textnormal{d}}
\newcommand{\ve}[1]{\textnormal{vec}(#1)}


% instälningar för figurtexter
%\usepackage[margin=3ex,font=small,labelfont=bf,labelsep=endash]{caption}
\usepackage[font={small,it}]{caption}
\usepackage[labelfont={normal,bf}]{caption}
\usepackage[margin=3ex]{caption}

% mailadresser som hyperlänkar
\newcommand{\mail}[1]{\href{mailto:#1}{\nolinkurl{#1}}}
% Spara författare och titel
\let\oldAuthor\author
\renewcommand{\author}[1]{\newcommand{\myAuthor}{#1}\oldAuthor{#1}}
\let\oldTitle\title
\renewcommand{\title}[1]{\newcommand{\myTitle}{#1}\oldTitle{#1}}

% Hyperlänkar
\hypersetup{
  colorlinks   = true, %Colours links instead of ugly boxes
  urlcolor     = black, %Colour for external hyperlinks
  linkcolor    = black, %Colour of internal links
  citecolor   = black  %Colour of citations
}

\definecolor{mygreen}{rgb}{0,0.6,0}
\definecolor{mygray}{rgb}{0.5,0.5,0.5}
\definecolor{mymauve}{rgb}{0.58,0,0.82}

\lstset{ %
  backgroundcolor=\color{white},   % choose the background color; you must add \usepackage{color} or \usepackage{xcolor}
  basicstyle=\footnotesize\tt,        % the size of the fonts that are used for the code
  breakatwhitespace=false,         % sets if automatic breaks should only happen at whitespace
  breaklines=true,                 % sets automatic line breaking
  captionpos=t,                    % sets the caption-position to bottom
  commentstyle=\color{mygreen},    % comment style
  deletekeywords={...},            % if you want to delete keywords from the given language
  escapeinside={\%*}{*)},          % if you want to add LaTeX within your code
  extendedchars=true,              % lets you use non-ASCII characters; for 8-bits encodings only, does not work with UTF-8
  frame=single,	                   % adds a frame around the code
  keepspaces=true,                 % keeps spaces in text, useful for keeping indentation of code (possibly needs columns=flexible)
  keywordstyle=\color{blue},       % keyword style
  language=C,                 % the language of the code
  otherkeywords={...},           % if you want to add more keywords to the set
  numbers=left,                    % where to put the line-numbers; possible values are (none, left, right)
  numbersep=5pt,                   % how far the line-numbers are from the code
  numberstyle=\tiny\color{mygray}, % the style that is used for the line-numbers
  rulecolor=\color{black},         % if not set, the frame-color may be changed on line-breaks within not-black text (e.g., comments (green here))
  showspaces=false,                % show spaces everywhere adding particular underscores; it overrides 'showstringspaces'
  showstringspaces=false,          % underline spaces within strings only
  showtabs=false,                  % show tabs within strings adding particular underscores
  stepnumber=2,                    % the step between two line-numbers. If it's 1, each line will be numbered
  stringstyle=\color{mymauve},     % string literal style
  tabsize=2,	                   % sets default tabsize to 2 spaces
  title=\lstname                   % show the filename of files included with \lstinputlisting; also try caption instead of title
}



\graphicspath{{./matlab/},{./images/},{./matlab/figs/}} % Söker också bilder i en undermapp figs.

%% DOCUMENT
%------------------------------------------------------------------%
\begin{document}
  \title{Wave Packet Dynamics}

  \author{
    Joar Svartholm - josv0150(\mail{josv0150@student.umu.se})\\
  }
  \date{\today}

\begin{titlepage}
  \maketitle
  \thispagestyle{fancy}
  \headheight 35pt
  \rhead{\small\today}
  \lhead{\small Department of Physics\\
    Umeå Universitet}

% State the aim of the experiment, what was measured, which techniques and methods were used, and the main result(s) and conclusion(s). Remember that the abstract should be understandable on its own, and you can thereby not refer to equations/figures/tables in the report. You should also not use references, since the information in the abstract should be available in the actual report.
\begin{abstract}
  \noindent According to the Heisenberg principle, it is impossible to know the exakt position and and momentum of a quantum particle at the same time. This motivates the use of a wavefunction describing the probability of a particle at a given position. When simulating the dynamics of such particle, that is by solving the time dependent Schrödinger equation, one must know the wavefunction all over space at each time step. This makes the problem extremely difficult and computationally heavy. In this report, a quantum particle is simulated using two different numerical methods, the split operator method and the second-order differencing method. The particle is simulated as a free particle, hitting a potential well and a potential barrier. For the well and the barrier, the reflection and transmission coefficient was computed and compared to analytical expressions. The results of this was in good accordance to what was expected. The numerical solutions seemed to undershoot the transmission coefficient near the resonating energies, that is for energies close to the eigenenergies of the potential well. This could be explained as the simulated particles contained a distribution of wavenumbers and the analytical expressions was basen on monochromatic waves.
  The two numerical methods was then compared with respect to computation time and the results showed that the split operator method was faster. There is however room for improvements on the code for the second-order differencing method which is why more testing must be done before one can conclude this for sure.

\end{abstract}

  % Ändra till rätt namn m.m.
  \cfoot{Numerical Methods in Physics\\
  Supervisor:Claude Dion }

\end{titlepage}


\newpage
\pagestyle{fancy}
\headheight 30pt
\rhead{\small \myTitle}
\lhead{\small \myAuthor \today}
\cfoot{\thepage}

% Innehåll
\tableofcontents
\newpage

\section{Introduction}

Quantum particles does not behave like ordinary particles. This is due to the fact that their motion cannot simply be determined by the position and velocity of the particle, that is, because their position is not well defined. According to the Heisenberg principle, it is impossible to determine the exact position and momentum of a particle at the same time. The position can however be modeled by a probability density, spanning all over space. This makes it possible only to find analytic solutions to the dynamics of a quantum particle for only a few sets of potentials and initial conditions. This is why numerical methods are of high importance when studiyng quantum effects. It is possible to model and simulate these by solving the time dependent Schrödinger equation (TDSE) numerically using high level algorithms. The complexity of this situation is that since the position is described by a probability density, a wavefunction, one must store and manipulate data from all over space in each time step in comparison to ordinary particles where only the position an momentum is needed. This makes the problem of solving TDSE non-trivial and only high performance computers able to finish the task within reasonable time.

In this report, TDSE will be solved for a free particle, a particle hitting a potential well and a potential barrier and the quantum effects that occur will be studied. In the end two different numerical methods of solving TDSE for the potential barrier will be compared in means of computational time.

\newpage
\section{Theory}

The dynamics of a quantum particle is described by the time dependent Schrödinger equation

\begin{equation}
  \label{eq:TDSE}
  i \hbar \frac{\dn}{\dn t}\psi(x,t) = \hat{H}\psi(x,t) \tag{TDSE}
\end{equation}

where $\hbar$ is planck's constant, $\psi$ is the wavefunction of the particle, $i = \sqrt{-1}$ and $\hat{H}$ is the Hamiltonian

\begin{equation}
  \label{eq:ham}
  \hat{H} = \hat{T} + \hat{V} = - \frac{\hbar^2}{2m}\frac{\dn^2}{\dn x^2} + V(x),
\end{equation}

the sum of the kinetic and potential operator. In this equation, $m$ is the mass of the particle and $V(x)$ is the potential. The naive solution to Eq. \eqref{eq:TDSE} is

\begin{equation}
  \label{eq:TDSEsol}
  \psi(x,t+\Delta t) \approx e^{-\frac{i\Delta t}{\hbar}\hat{H}} \psi(x,t)
\end{equation}

for any small step size $\Delta t$. The problem with this naive solution is that taking the exponential of an operator is non-trivial.

\subsection{Split operator method}

One way to get around this is by noting that the kinetic operator includes a second derivative in position space but a a simple multiplication in momentum space. Namely that

\begin{equation*}
  e^{\frac{i \Delta t \hbar}{2m}\frac{\dn^2}{\dn x^2}}\psi(x,t) \longleftrightarrow e^{- \frac{i \Delta t }{2m\hbar}p^2}\psi(p,t)
\end{equation*}

This motivates the splitting of the hamilton operator into two parts, taking the potential operator in position space and the kinetic operator in ordinary space. The final time stepping can thus be written as

\begin{equation}
  \label{eq:split}
  \psi(x,t+\Delta t) = e^{-\frac{i\Delta t}{2\hbar}\hat{V}}\mathcal F^{-1} e^{-\frac{i\Delta t}{\hbar}\hat{T_p}}\mathcal F e^{-\frac{i\Delta t}{2\hbar}\hat{V}} \psi(x,t)
\end{equation}

where $\hat{T_p}$ denotes the kinetic operator in momentum space and $\mathcal F$ and $\mathcal F^{-1}$ denotes the direct and inverse fourier transform to go from position space to momentum space and vice versa. Note that the potential operator is also split into two parts in order to make the operation unitary and thus reversible.

\subsection{Second-order differencing}

\subsection{The free particle}

In order to study the numerical algorithms for something more interesting it is good to first try them on something with a known solution. Consider the initial gaussian wavefunction with central wavenumber $k_0$ and width $2\sigma_0^2$

\begin{equation}
  \label{eq:init}
  \psi(x,0) = \left( \frac{1}{\pi \sigma_0^2} \right)^{1/4} e^{ik_0x}e^{-(x-x_0)^2/2\sigma_0^2}.
\end{equation}

The analytical solution to Eq. \eqref{eq:TDSE} for this initial condition using $V(x)=0$ is

\begin{equation}
  \label{eq:free}
  \psi(x,t) = \left(\frac{\sigma_0^2}{\pi}\right)^{1/4} \frac{e^{i\phi}}{(\sigma_0^2 + it)^{1/2}} e^{ik_0x} exp \left[ - \frac{(x-x_0-k_0t)^2}{2\sigma_0^2 + 2it}\right]
\end{equation}

where $\phi \equiv -\theta -k_0^2t/2$ and $\tan \theta = t/\sigma_0$. By comparing this analytial solution to the one obtained by the numerical solution one can determine the validity and and stability of the numerical algorithm.

\subsection{The potential well}

Changing the potential to

\begin{equation}
  \label{eq:pot-well}
  V(x) = \begin{cases}
    -V_0 & \quad , |x|<a \\
    0 & \quad, \text{else}
  \end{cases}
\end{equation}

yields a more interesting case, with $V_0$ being a positive constant. Outside of the well the wave will propagate as a free particle (Eq. \eqref{eq:free}) but when it hits the well it will either transmit through or reflect back. Since the wavefunction corresponds to a probability it can be shown that the probability of transmittance and reflectance will depend on the energy of the wave, and hence depend on the central wavenumber $k_0$. Namely that

\begin{equation}
  \label{eq:well-T}
  T = \frac{1}{ \left( 1+ [V_0^2 /4 E (E+V_0)]\sin^2(2a\sqrt{2m(E+V_0)/\hbar^2})    \right)  },
\end{equation}

where $E =\hbar^2 k_0^2/2m$ is the average energy of the particle. Now $T$ is the probability of the particle being transmitted through the well and similarly $R=1-T$ is the probability of reflectance. Looking at Eq. \ref{eq:well-T} on can see that it will resonate with the potential well and have maximum when

\begin{equation}
  \label{eq:resonance}
  2a\sqrt{\frac{2m(E+V_0)}{\hbar^2}} = n \pi, \quad n= 1,2,...
\end{equation}

This could be viewed as standing waves in the potential well, making it transparent to the wave packet.


\subsection{The potential barrier}

Changing the sign of $V_0$ in Eq. \eqref{eq:pot-well} yields a potential barrier. Now the intuitive solution would be that the the quantum particle would bounce on the barrier but in quantum mechanics, there is always a probability of transmittance, regardless of the energy of the particle. This is called tunneling. If the energy of the particle is less than the energy of the barrier, the transmission coefficient will be the same as Eq. \eqref{eq:well-T} but with a change of sign from $V_0$ to $-V_0$. If the energy is larger than the barrier height, the transmission coefficient will be

\begin{equation}
  \label{eq:barr-T}
  T = \frac{1}{ \left( 1+ [V_0^2/4E(V_0-E)]\sinh^2 (2a\sqrt{2m(V_0-E)/\hbar^2})   \right)  }.
\end{equation}

The wavenumber corresponding to the energy of the potential barrier is $k_V = \sqrt{2mV_0/\hbar^2}$.

\section{Method}
\label{sec:method of stuff}

In order to study the quantum particle using the methods in Sec. \ref{sec:Theory} the program called \verb|wavepacket| was used \cite{wavepacket}. First, the initial wavefunction was implemented together with a zero potential and the wave was simulated for $6.8 \cdot 10^{-3}$s using $k_0 = 200$ and the resulting wave was analyzed. Next, the potential was replaced with a potential well. For this case simulations was done for different inital wavenumbers from 250 to 350 in steps of 10 in order to find a relation of the reflection and transmission coefficients and thus be able to verify Eq. \eqref{eq:well-T}. The reflection and transmission coefficients was computed as

\begin{equation}
  \label{eq:RTnumerical}
  \begin{split}
    R &= \int_{-\infty}^{-a}|\psi(x,t)|^2\dn x \approx 0.5\psi(x=-8,t) + \sum_{i=1}^{x(i)<-a} \psi(x=-8+i\Delta x,t) + 0.5\psi(x=-a,t)\\
    T &= \int_{a}^{\infty}|\psi(x,t)|^2\dn x \approx 0.5\psi(x=a,t) + \sum_{x(i)>a}^{x(i)<8} \psi(x=a+i\Delta x,t) + 0.5\psi(x=8,t)\\
  \end{split}
\end{equation}

The final coefficients of each simulation was then used to plot $R,T$ against the wavenumber.

The same study was then done with the potential changed to a barrier instead of a well.

As a final study the numerical algorithm was changed and the program \verb|diffScheme| was used. The same results as with the potential barrier was produced and the two algorithms was compared by the means of excecution time. In order to make a fair comparison both programs produced the same output and was set to the same simulated time in means of seconds.

\subsection{wavepacket.c}

The program \verb|wavepacket| is a simulation program for quantum particles using the split-operator method described in Sec. \ref{sec:splitOp}. The program was used with different potentials yielding different executable files. There three potentials can be compiled using either of the commands

\begin{itemize}
  \item \verb|make freepart|
  \item \verb|make potwell|
  \item \verb|make barrier|
\end{itemize}

for the free particle, potential well and potential barrier correspondingly. The program uses \verb|FFTW| in order to compute the fourier transforms so this library must be installed in order to compile the program correctly.

The simulations are then executed by adding and argument file containing the parameters (preferrably the template file in the appendix \ref{App:input}) and an argument with the initial wavenumber. If the input file is found in a folder called \verb|input_files| then a simulation for the potential well using $k_0=200$ could be excetuted using the line

\verb|./potwell input_files/pot_well_template.in 200|

\subsection{waveDiff.c}
\label{sec:diffScheme}

The program implementing the second-order differencing scheme is called \verb|waveDiff| and can be compiled by the command \verb|make diffScheme|. No external libraries was used in the implementation of this code. The same datatypes and general funcitons (e.g. the routine for reading parameters and printing results) as in \verb|wavepacket| was used in order to be able to compare the computation times. A similar command can be used to execute a simulation using this program. Note however that the time step used must be about $5\cdot 10^{-8}$ in order to achieve convergence.

\section{Results}

\begin{figure}[H]
  \centering
  \includegraphics[width=0.8\textwidth]{../wavepacket/figs/free-part-norm2}
  \caption{Numerical and analytic solution of the time dependent schrödinger equation for a free particle.}
  \label{fig:free-norm}
\end{figure}

\begin{figure}[H]
  \centering
  \includegraphics[width=0.8\textwidth]{../wavepacket/figs/free-part-im}
  \caption{Imaginary part of the solution in Fig. \ref{fig:free-norm}.}
  \label{fig:free-imag}
\end{figure}

\begin{figure}[H]
  \centering
  \includegraphics[width=0.8\textwidth]{../wavepacket/figs/pot-well-wave}
  \caption{Numerical solution of the time dependent schrödinger equation of the potential well.}
  \label{fig:well-wave}
\end{figure}

\begin{figure}[H]
  \centering
  \includegraphics[width=0.8\textwidth]{../wavepacket/figs/pot-well-RT}
  \caption{Reflection and transmission coefficient of the particle in Fig. \ref{fig:well-wave}.}
  \label{fig:well-RT}
\end{figure}

\begin{figure}[H]
  \centering
  \includegraphics[width=0.8\textwidth]{../wavepacket/figs/pot-well-RTk0}
  \caption{Reflection and transmission coefficient for different center wave numbers used.}
  \label{fig:well-k0}
\end{figure}

\begin{figure}[H]
  \centering
  \includegraphics[width=0.8\textwidth]{../wavepacket/figs/pot-barrier-wave}
  \caption{Numerical solution of the time dependent schrödinger equation of the potential barrier.}
  \label{fig:barrier-wave}
\end{figure}

\begin{figure}[H]
  \centering
  \includegraphics[width=0.8\textwidth]{../wavepacket/figs/pot-barrier-RT}
  \caption{Reflection and transmission coefficient for the particle in Fig. \ref{fig:barrier-wave}}
  \label{fig:barrier-RT}
\end{figure}

\begin{figure}[H]
  \centering
  \includegraphics[width=0.8\textwidth]{../wavepacket/figs/pot-barrier-RTk0}
  \caption{Reflection and transmission coefficient for different center wavenumbers used. $k_V$ is the wavenumber corresponding to the potential barrier height.}
  \label{fig:barrier-k0}
\end{figure}

\section{Conclusion}

As a conclusion one can say that the numerical results for simulating a quantum particle performs well in both predicting the dynamics of it and verifying quantum effects which are otherwise only proven analytically for simple cases and experimentally. The results showed that the wavepacket can be reflected when meeting a potential well and, by resonance, it can also be almost perfectly transmitted. The numerical transmission coefficient matched the analytical expression quite well  but undershot the peak values in transmission. This is due to that the analytical expression is basen on a monochromatic wave and the simulated wave was a wavepacket consisting of waves with a finite bandwidth. Similar results was obtained for the potential barrier in which also the tunneling effect could be proven. This means that a quantum particle meeting a more energetic potetial barrier might still pass through it.

When comparing the two algorithms split operator method and second-prder differencing method the split operator method was a lot faster with respect to computational time. If this result is perfectly valid is not clear since there might be some code optimization still possible to do on the second-order differencing code. In order to be able to draw more conclusions regarding this one must first compare different datatypes and try to make the code more efficient. This report showed however that the split operator method is more likely to be faster.


%\addcontentsline{toc}{section}{R References}
\bibliographystyle{unsrt}
\begin{thebibliography}{}
  \bibitem{sakurai} J. J. Sakurai and J. Napolitano, Modern Quantum Mechanics (Cambridge
  University Press, Cambridge, 2017

  \bibitem{wavepacket}C. M. Dion, A. Hashemloo, and G. Rahali, Program for quantum wavepacket
dynamics with time-dependent potentials, Comput. Phys. Commun.
185, 407 (2014).

\end{thebibliography}

\appendix
\section{Codes}
\label{sec:Appendix}

\lstinputlisting{code/input_files/pot_well_template.in}
\lstinputlisting{code/hamiltonian_operator.c}


\end{document}
